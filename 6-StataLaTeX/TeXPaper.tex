\documentclass[11pt]{article} % use larger type; default would be 10pt
\usepackage[utf8]{inputenc} % set input encoding (not needed with XeLaTeX)
\usepackage{pslatex}
\usepackage{standalone} %required by outreg2 in Stata
\usepackage{dcolumn} %required by stargazer in R

\title{Your Paper That's So Transparent And Reproducible You'll Get a Tenure Track Position}
\author{Garret Christensen}
%\date{} % Activate to display a given date or no date (if empty),
         % otherwise the current date is printed 

\begin{document}
ECHO is on.
\newcommand{\meanprice}{6165.256756756757} 
\newcommand{\stddevprice}{2949.5} 

%Just load the file full of scalars at the beginning
%Call them later when you need them
\maketitle

\section{Wait for it}

This is where I introduce you to my amazingness.
I'm going to do a little thing where I include numerical output from Stata.

The mean of the price variable is $\meanprice$ and its standard deviation (rounded to one decimal point) is $\stddevprice$

\subsection{Wait for it.}

More text, and ... a table! 

\begin{table}
\caption{Made Automatically in Stata}
\documentclass[]{article}
\setlength{\pdfpagewidth}{8.5in} \setlength{\pdfpageheight}{11in}
\begin{document}
\begin{tabular}{lccc} \hline
 & (1) & (2) & (3) \\
VARIABLES & free\_chl\_yn & free\_chl\_yn & free\_chl\_yn \\ \hline
 &  &  &  \\
treatw & 0.364*** & 0.364*** & 0.365*** \\
 & (0.040) & (0.044) & (0.044) \\
english &  &  & 0.034 \\
 &  &  & (0.063) \\
kiswahili &  &  & -0.011 \\
 &  &  & (0.076) \\
Constant & 0.013 & 0.013 & -0.003 \\
 & (0.027) & (0.009) & (0.046) \\
 &  &  &  \\
Observations & 284 & 284 & 284 \\
 R-squared & 0.223 & 0.223 & 0.224 \\ \hline
\multicolumn{4}{c}{ Standard errors in parentheses} \\
\multicolumn{4}{c}{ *** p$<$0.01, ** p$<$0.05, * p$<$0.1} \\
\end{tabular}
\end{document}
 %created by StataLaTeX.do
\end{table}
%input the table you generated in Stata with outreg2 or estout
%This is a relatively simple 2-click workflow
%One click in Stata to run all your code and generate the results
%and a second click to compile and include those results into your paper.

%The concern is that with a lot of tables, you can forget which script file generated
%which table. Hence the comment above next to the input statement.



\end{document}
