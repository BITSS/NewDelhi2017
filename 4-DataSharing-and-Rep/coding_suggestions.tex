\documentclass[12pt] {article}
\usepackage{fullpage}
\begin{document}

\section{General Workflow
Suggestions:}\label{general-workflow-suggestions}
Here we offer some specific workflow organization suggestions that should be valid regardless of code or operating system.
\begin{itemize}
\item
  Do not use spaces in directory or file names, as it complicates referring to them in certain software.
\item
  Use ``naming directories'', .i.e.~a directory beginning with ``-'' (so
  that it will appear first alphabetically) inside each directory to
  explain the contents of the above directory.
\item
  Add name, date, and describe contents, as well as updates, to all
  scripting files.
\item
  Keep a daily research log, i.e. a detailed written diary of what research is done on a given day. You'll be surprised how often this will be useful to answer questions about whether you ran a certain test or not, when you did it, and what you called the file.
  
\item
    Make sure that your script files are self-contained. That is, don't write a program that only works if you run a group of other files previously in a specific order and then leave things a certain precarious way. You should be able to run intermediate steps of the workflow without disrupting things downstream.
 
\item
  You can never comment too much.
\item
  Indent your code
\item
    Once you post/distribute code or data, any changes at all require a new file name.
\item
  Separate your cleaning and analysis files; don't make any new variables
  that need saving (or will be used by multiple analysis
  files) in an analysis file---it is better to only create the variables once so you know they're the identical when used in different analysis files.
\item
  Never name a file ``final'' because it won't be.
\item
  Name binary variables ``male'' instead of ``gender'' so that the name is more informative. 
\item
  Use a prefix such as x\_ or temp\_ so you know which files can easily
  be deleted.
\item
  Never change the contents of a variable unless you give it a new name.
\item
  Every variable should have a label.
\end{itemize}

\subsection{Stata-specific
Suggestions}\label{stata-specific-suggestions}

\begin{itemize}
\item
  Use the full set of missing values available to you (``.a''-``.z'', not exclusively
  ``.'') in order to distinguish between ``don't know'' and ``didn't ask'' or other distinct reasons for missing data.
\item
  Make sure code always produces same result---if you use anything randomly generated, set the seed. When sorting or merging, you need to be sure to uniquely specify observations, because if you don't, Stata does something arbitrary and not repeatable. So instead of just sorting or merging on `ID' when there are multiple observations per ID, sort by `ID' and `name.' You can use the `duplicates' command to test whether the \textit{varlist} you use uniquely indentifies observations. The `sort, stable' command can be used, though it is slower.
\item
  Use the `version' command in your .do file to ensure that other researchers who run your code with a newer version of Stata get the same results. 
\item
  Don't use abbreviations for variables (which may become unstable after
  adding variables) or commands (beyond reason)
\item
  It's a common piece of advice to avoid using global macros, and use locals instead. (Since globals can be accessed across different functions or spaces, they can create contradictions or inconsistent dependencies.) However, in Stata, it's possible to safely use them to define directory paths so collaborators can work across different computers. An alternative to globals is to use: `capture cd C:/Users/garret/Documents/Researchproject' and have one line for each user/computer.
  
\item
  Use locals for varlists to ensure that long lists of variables include the same variables whenever intended.
\item
  Use computer-stored versions of numerical output instead of manually typing in numbers or copying and pasting. For example, instead of copying and pasting the mean after a `summ' command, refer to `r(mean)'. Use the `return list' command to see a full list of stored values after a regular command and the `ereturn list' after estimation commands.
\item
  If you have a master .do file that calls other .do files, which each have their own .log file, you can run
  multiple log files at the same time (so you have a master .log file)
\item
  Use the `label data' and `notes' commands to label datasets and help yourself and other researchers easily identify the contents.
\item
  Use the `notes' command for variables as well for identifying information that is too long for the variable label.
\item
  Use the `datasignature' command to generate a hash or checksum and help ensure that data is
  the same as before.
\item
  In addition to labeling your variables, you should also use value labels for all categorical variables. Include the numerical value in the label, however, since without it, it can be hard to tell what numerical value is actually meant by a given category. `numlabel , add' is your friend.
\item
  Even though Stata is case sensitive, don't use capital letters in variable names since not all software packages
  are case sensitive.
\item
  Make your files as non-proprietary as possible (use the `saveold'
  command to enable those with earlier versions to use your data. This
  is why trusted repositories are so useful--they'll do this for you.)

\item Use the `tempfile' and `tempvar' commands to save space and not clutter up datasets and folders with temporary files and variables.

   
  
\end{itemize}
\end{document}